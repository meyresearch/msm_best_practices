%%%%%%%%%%%%%%%%%%%%%%%%%%%%%%%%%%%%%%%%%%%%%%%%%%%%%%%%%%%%
%%% LIVECOMS ARTICLE TEMPLATE FOR BEST PRACTICES GUIDE
%%% ADAPTED FROM ELIFE ARTICLE TEMPLATE (8/10/2017)
%%%%%%%%%%%%%%%%%%%%%%%%%%%%%%%%%%%%%%%%%%%%%%%%%%%%%%%%%%%%
%%% PREAMBLE
\documentclass[9pt,bestpractices]{livecoms}
% Use the 'onehalfspacing' option for 1.5 line spacing
% Use the 'doublespacing' option for 2.0 line spacing
% Use the 'lineno' option for adding line numbers.
% Use the "ASAPversion' option following article acceptance to add the DOI and relevant dates to the document footer.
% Use the 'pubversion' option for adding the citation and publication information to the document footer, when the LiveCoMS issue is finalized.
% The 'bestpractices' option for indicates that this is a best practices guide.
% Omit the bestpractices option to remove the marking as a LiveCoMS paper.
% Please note that these options may affect formatting.

\usepackage{lipsum} % Required to insert dummy text
\usepackage[version=4]{mhchem}
\usepackage{siunitx}
\DeclareSIUnit\Molar{M}
\usepackage[italic]{mathastext}
\graphicspath{{figures/}}

%%%%%%%%%%%%%%%%%%%%%%%%%%%%%%%%%%%%%%%%%%%%%%%%%%%%%%%%%%%%
%%% IMPORTANT USER CONFIGURATION
%%%%%%%%%%%%%%%%%%%%%%%%%%%%%%%%%%%%%%%%%%%%%%%%%%%%%%%%%%%%

\newcommand{\versionnumber}{0.1}  % you should update the minor version number in preprints and major version number of submissions.
\newcommand{\githubrepository}{\url{https://github.com/meyresearch/msm_best_practices}}  %this should be the main github repository for this article

%%%%%%%%%%%%%%%%%%%%%%%%%%%%%%%%%%%%%%%%%%%%%%%%%%%%%%%%%%%%
%%% ARTICLE SETUP
%%%%%%%%%%%%%%%%%%%%%%%%%%%%%%%%%%%%%%%%%%%%%%%%%%%%%%%%%%%%
\title{Best practices for Markov state model creation [Article v\versionnumber]}

\author[1,2]{Robert E. Arbon}
\author[1]{Antonia S. J. S. Mey}
\author[3]{Brooke E. Husic}
\author[4]{Sukrit Singh}
\author[5]{Sonya M. Hanson}
% \author[2\authfn{1}\authfn{4}]{Firstname Initials Surname}
% \author[2*]{Firstname Surname}
\affil[1]{EaStCHEM School of Chemistry, David Brewster Road, Joseph Black Building, The King’s Buildings, Edinburgh, EH93FJ, UK}
\affil[2]{ReDesign Science, New York, NY, USA}
\affil[3]{Princeton University}
\affil[4]{Memorial Sloan-Kettering Cancer Center}
\affil[5]{Flatiron Institute}

\corr{antonia.mey@ed.ac.uk}{ASJSM}  % Correspondence emails.  FMS and FS are the appropriate authors initials.
% \corr{email2@example.com}{FS}

\orcid{Robert E. Arbon}{0000-0001-6163-3029}
\orcid{Antonia S. J. S. Mey}{0000-0001-7512-5252}

\contrib[\authfn{1}]{These authors contributed equally to this work}
\contrib[\authfn{2}]{These authors also contributed equally to this work}

\presentadd[\authfn{3}]{Department, Institute, Country}
\presentadd[\authfn{4}]{Department, Institute, Country}

\blurb{This LiveCoMS document is maintained online on GitHub at \githubrepository; to provide feedback, suggestions, or help improve it, please visit the GitHub repository and participate via the issue tracker.}

%%%%%%%%%%%%%%%%%%%%%%%%%%%%%%%%%%%%%%%%%%%%%%%%%%%%%%%%%%%%
%%% PUBLICATION INFORMATION
%%% Fill out these parameters when available
%%% These are used when the "pubversion" option is invoked
%%%%%%%%%%%%%%%%%%%%%%%%%%%%%%%%%%%%%%%%%%%%%%%%%%%%%%%%%%%%
\pubDOI{10.XXXX/YYYYYYY}
\pubvolume{<volume>}
\pubissue{<issue>}
\pubyear{<year>}
\articlenum{<number>}
\datereceived{Day Month Year}
\dateaccepted{Day Month Year}

%%%%%%%%%%%%%%%%%%%%%%%%%%%%%%%%%%%%%%%%%%%%%%%%%%%%%%%%%%%%
%%% ARTICLE START
%%%%%%%%%%%%%%%%%%%%%%%%%%%%%%%%%%%%%%%%%%%%%%%%%%%%%%%%%%%%

\begin{document}

\begin{frontmatter}
\maketitle

\begin{abstract}
Markov state models are a useful tool for learning the dynamic processes of chemical systems from molecular dynamics simulations. 
\end{abstract}

\end{frontmatter}


\section{Introduction - Surkit/Brooke}
\begin{enumerate}
    \item Discuss important role of protein folding and dynamics in understanding disease and basic biology
    \item To capture important states of protein folding and dynamics many models have been made
    \item Discuss major models used to study biomolecular dynamics
    \item 2-state models, experimental models
    \item One of the most successful ones is the Markov State Model (MSM)
    \item An MSM is a network representation of a molecule’s free energy landscape
    \item Sampling of simulations at long timescales (FAH) has bolstered the power of MSMs to identify all kinds of interesting dynamics and motions
    \item Examples of MSM success stories
    \item Slow timescales are still difficult to capture however even with this kind of exascale sampling power
    \item Construction of an MSM is a multi-step non trivial process
    \item Describe the construction process of an MSM broadly
    \item Furthermore, the robustness of these models, and their hyperparameters, changes as you go towards larger and more complex biological systems
    \item In this review, we summarize current approaches to MSM construction, validation, and usage, and provide a unified resource for the tools and methods available to one during the process of MSM construction. 
    \item This review will first provide an overview of Markov Chain terminology and the underlying theory. 
    \item We will also list and discuss existing toolkits used to generate MSMs from existing trajectory datasets, alongside the features and strengths of each toolkit.
    \item We will follow this up with a discussion of hyperparameter selection, uncertainty calculation, and validation of MSMs.
    \item Lastly, we will provide examples where MSMs have been used to great effect in analyzing biological systems in conjunction with experiment. We will provide an overview of how experiments can synergize well with Markov State Models, as well as what parameters were used to achieve robust, synergistic models
\end{enumerate}



\section{Science topics - Rob/Toni}


\section{Prerequisits - Toni/Sonya}

Here you would identify prerequisites/background knowledge that are assumed by your work and your checklist which you view as critical, ideally giving links to good sources on these topics.
Checklists are normally focused on errors made by users with training and experience in molecular simulations, so you can assume a basic familiarity with the fundamentals of molecular simulations.

\section{Overview of approaches - Brooke/Sonya}


\section{Selecting hyperparameters - Sonya/Brooke}

\section{Measuring uncertainty  - Rob/Sonya}

\section{Validating results - Sukrit/Rob}

\section{Checklist}
Here we use a full-page checklist with multiple sections, so it will appear on a separate page of the sample PDF.
Other checklist formats are possible, as shown in the sample \texttt{sample-document.tex} in \url{github.com/livecomsjournal/article_templates/templates}.

Your checklist should include a succinct list of steps that people should follow when carrying out the task in question.
This is provided to ensure certain basic standards are followed and common but critical major errors are avoided.
Note that a checklist is not intended to cover \emph{all} important steps, but rather focus on the most common reasons for failure or incorrect results, or issues which are particularly crucial.


% This provides a checklist which
% - spans a full page
% - consists of multiple sub-checklists
% - exists on a separate page
% This style of checklist will be especially helpful if you want to encourage readers to print and use your checklist in practice, as they
% can easily print it without also printing other material from your manuscript. However, other styles of checklist are also possible (below).
\begin{Checklists*}[p!]

\begin{checklist}{First list}
\textbf{You can easily make full width checklists that take a whole page}
\begin{itemize}
\item Items in your checklist can and should reference sections where the cited issues are discussed in detail, such as Section~\ref{sec:reference_this}
\item Also remember
\item And finally
\end{itemize}
\end{checklist}

\begin{checklist}{Second list}
\textbf{This is some further description.}
\begin{itemize}
\item First thing
\item Also remember
\item And finally
\end{itemize}
\end{checklist}

\begin{checklist}{Third list}
\textbf{This is some further description.}
\begin{itemize}
\item First thing
\item Also remember
\item And finally
\end{itemize}
\end{checklist}

\begin{checklist}{Fourth list}
\textbf{This is some further description.}
\begin{itemize}
\item First thing
\item Also remember
\item And finally
\end{itemize}
\end{checklist}

\end{Checklists*}





\section{Rationale}

Your Rationale section, or sections, can follow or precede your checklist (we expect that often, following the checklist will be preferable) and provide the necessary rationale for the checklist, and act as more complete \emph{best practices} description.
This should include 1) significant detail as to the possible alternative ways to accomplish a given task, 2) description of advantages and disadvantages of the various approaches, and 3) significant literature documentation about reasons for choices.

\subsection{Algorithms and Pseudocode}
\label{sec:reference_this}

The \texttt{algpseudocode} and \texttt{algorithms} packages is loaded by the document class. \ALG{euclid} was taken directly from the package documentation. (Please do not load \texttt{algorithm2e}; it's not compatible with \texttt{algpseudocde} nor \texttt{algorithms}!)

\begin{algorithm}
\caption{Euclid's algorithm}\label{alg:euclid}
\begin{algorithmic}%[1]  %% uncomment to enable line numbers
\Procedure{Euclid}{$a,b$}\Comment{The g.c.d. of a and b}
   \State $r\gets a\bmod b$
   \While{$r\not=0$}\Comment{We have the answer if r is 0}
      \State $a\gets b$
      \State $b\gets r$
      \State $r\gets a\bmod b$
   \EndWhile\label{euclidendwhile}
   \State \textbf{return} $b$\Comment{The gcd is b}
\EndProcedure
\end{algorithmic}
\end{algorithm}





\section{Author Contributions}
%%%%%%%%%%%%%%%%
% This section mustt describe the actual contributions of
% author. Since this is an electronic-only journal, there is
% no length limit when you describe the authors' contributions,
% so we recommend describing what they actually did rather than
% simply categorizing them in a small number of
% predefined roles as might be done in other journals.
%
% See the policies ``Policies on Authorship'' section of https://livecoms.github.io
% for more information on deciding on authorship and author order.
%%%%%%%%%%%%%%%%

(Explain the contributions of the different authors here)

% We suggest you preserve this comment:
For a more detailed description of author contributions,
see the GitHub issue tracking and changelog at \githubrepository.

\section{Other Contributions}
%%%%%%%%%%%%%%%
% You should include all people who have filed issues that were
% accepted into the paper, or that upon discussion altered what was in the paper.
% Multiple significant contributions might mean that the contributor
% should be moved to authorship at the discretion of the a
%
% See the policies ``Policies on Authorship'' section of https://livecoms.github.io for
% more information on deciding on authorship and author order.
%%%%%%%%%%%%%%%

(Explain the contributions of any non-author contributors here)
% We suggest you preserve this comment:
For a more detailed description of contributions from the community and others, see the GitHub issue tracking and changelog at \githubrepository.

\section{Potentially Conflicting Interests}
%%%%%%%
%Declare any potentially competing interests, financial or otherwise
%%%%%%%

Declare any potentially conflicting interests here, whether or not they pose an actual conflict in your view.

\section{Funding Information}
%%%%%%%
% Authors should acknowledge funding sources here. Reference specific grants.
%%%%%%%
FMS acknowledges the support of NSF grant CHE-1111111.

\section*{Author Information}
\makeorcid


\section{Bibliography instructions}

\begin{enumerate}
    \item Commit Overleaf document to Github. 
    \item Upload refs to Zotero group. 
    \item Check for duplicates (sort by name).  \textbf{delete YOUR duplicate version}. 
    \item Download whole group as a bib file. 
    \item Upload to Overleaf. 
    \item Commit overleaf document to Github. 
\end{enumerate}

please make sure when you commit work there are no key errors or any problems with the bibliography.  This way we can catch errors as the arise. 

\bibliography{bibliography}

%%%%%%%%%%%%%%%%%%%%%%%%%%%%%%%%%%%%%%%%%%%%%%%%%%%%%%%%%%%%
%%% APPENDICES
%%%%%%%%%%%%%%%%%%%%%%%%%%%%%%%%%%%%%%%%%%%%%%%%%%%%%%%%%%%%

%\appendix


\end{document}
